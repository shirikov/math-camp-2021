\documentclass[11pt]{article}
\usepackage[latin1]{inputenc}
\usepackage[T1]{fontenc}
\usepackage[top=.8in, bottom=.8in, left=1in, right=1in]{geometry}
\renewcommand{\rmdefault}{pplx}
% \usepackage[sc]{mathpazo}
\usepackage{fourier}
\usepackage[OT1, euler-hat-accent]{eulervm}
\usepackage[usenames, dvipsnames, svgnames]{xcolor}
\usepackage{enumitem}
\usepackage{titling}
\usepackage[small, compact]{titlesec}
\setitemize[0]{leftmargin=*}
\usepackage{amsfonts, amsmath, amssymb}
\usepackage{multicol, multirow}
\usepackage{epsfig, subfigure, subfloat, graphicx}
\usepackage{anysize, indentfirst, setspace}
\usepackage{verbatim, rotating, xfrac}
\usepackage{gensymb}
\usepackage{caption, hanging}
\newcommand{\mc}[1]{\multicolumn{1}{c}{#1}}
%\parindent 0pt
%\setdefaultenum{a.}{i.}{A}{1}
%\setdefaultitem{}{\textperiodcentered}{}{}
\usepackage{booktabs}
\usepackage{dcolumn}
\usepackage{caption, hanging}
\usepackage{tikz}
\usetikzlibrary{shapes,arrows,backgrounds}
%\setdefaultenum{a.}{1)}{i.}{a.}
\parindent 0pt

\makeatletter
\newcommand{\distas}[1]{\mathbin{\overset{#1}{\kern\z@\sim}}}%
\newsavebox{\mybox}\newsavebox{\mysim}
\newcommand{\distras}[1]{%
  \savebox{\mybox}{\hbox{\kern3pt$\scriptstyle#1$\kern3pt}}%
  \savebox{\mysim}{\hbox{$\sim$}}%
  \mathbin{\overset{#1}{\kern\z@\resizebox{\wd\mybox}{\ht\mysim}{$\sim$}}}%
}
\makeatother

\title{\Large{\bf{\vspace{-100pt}Mathematics for Political Science \vspace{-15pt}}}}
\author{\large{Exercise Solution 3: Probability}}
\date{August 21\textsuperscript{st}, 2020}
\begin{document}
\maketitle

\hrule


\begin{enumerate}

\item 
$36 * 35 * 5 * 12 * 13 * 6 = 5896800$


\item 
\begin{enumerate}
\item \{Reagan, Bush31, Dole, Bush43, McCain, Romney, Perot, Nader\}
\item (graph - all the area outside of the circle around ``Rational Numbers'')
\item Whole Numbers, Natural Numbers
\item Natural Numbers
\item \{1,2,3,4,5,6\}
\item All the area around ``Whole Numbers'' except for the area representing ``Irrational Numbers''
\item $\{\emptyset\}$
\end{enumerate}


\item 
$A \cup A = A$ and $A \cap A = A$.  The intersection or union of any set with itself is itself.


\item 

\begin{small}
\begin{tabular}{c|l}
  & \\ \hline
\rule{0cm}{.5cm} 2 & $\frac{1}{16}$ \\
\rule{0cm}{.5cm} 3 & $\frac{2}{16} = \frac{1}{8}$ \\
\rule{0cm}{.5cm} 4 & $\frac{3}{16}$ \\
\rule{0cm}{.5cm} 5 & $\frac{4}{16} = \frac{1}{4}$ \\
\rule{0cm}{.5cm} 6 & $\frac{3}{16}$ \\
\rule{0cm}{.5cm} 7 & $\frac{2}{16} = \frac{1}{8}$ \\
\rule{0cm}{.5cm} 8 & $\frac{1}{16}$ \\
\end{tabular}
\end{small}


\item 
\begin{enumerate}
\item $\left(\frac{1}{2}\right)^6 = \frac{1}{64}$
\item $\binom{10}{3} \left(\frac{3}{13}\right)^3 \left(\frac{10}{13}\right)^7 \approx .235 $
\item $1 - \binom{9}{3} \binom{6}{3} \left(\frac{1}{6}\right)^9 \approx .99983 $
\end{enumerate}




\item 
\begin{enumerate}
\item $.53$
\item $\approx .444$
\item $\approx .396$
\item $\approx .333$
\item $\approx .595$
\end{enumerate}


\item $\frac{2}{11}$


\item 
\begin{enumerate}
\item $p(A|red) = \frac{1}{7} \approx .143$, $p(B|red) = \frac{6}{7} \approx .857$
\item $p(A|blue,blue) \approx .835$, $p(B|blue,blue)  \approx .165$ \\   $p(A|blue,blue, blue) \approx .919$, $p(B|blue,blue)  \approx .081$.
\item $p(A|red) = \approx .310$
\end{enumerate}


\item $\frac{7}{31} \approx .226$


\end{enumerate}

\vfill
\begin{center}
\small{Thanks to Sarah Bouchat, Michael DeCrescenzo, Brad Jones, and Dave Ohls for past years' materials.}
\end{center}

\end{document}



